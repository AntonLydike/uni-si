\documentclass{article}
\usepackage{microtype}
\usepackage[utf8]{inputenc} 
\usepackage[a4paper, total={6in, 9.6in}]{geometry}
\usepackage{enumitem}
\usepackage{amsmath}
\usepackage{amssymb}
\usepackage{fancyhdr}
\usepackage{xcolor}
\usepackage{tikz}
\usepackage{pgfplots}
\usepackage{svg}
\usepackage{graphicx}
\usepackage{listings}

\widowpenalties=4 10000 10000 150 0


% listings sonderzeichen
\lstset{
  literate={ö}{{\"o}}1
           {ä}{{\"a}}1
           {ü}{{\"u}}1
}
\lstdefinelanguage
   [x64]{Assembler}     % add a "x64" dialect of Assembler
   [x86masm]{Assembler} % based on the "x86masm" dialect
   % with these extra keywords:
   {morekeywords={CDQE,CQO,CMPSQ,CMPXCHG16B,JRCXZ,LODSQ,MOVSXD, %
                  POPFQ,PUSHFQ,SCASQ,STOSQ,IRETQ,RDTSCP,SWAPGS, %
                  rax,rdx,rcx,rbx,rsi,rdi,rsp,rbp, %
                  r8,r8d,r8w,r8b,r9,r9d,r9w,r9b, %
                  r10,r10d,r10w,r10b,r11,r11d,r11w,r11b, %
                  r12,r12d,r12w,r12b,r13,r13d,r13w,r13b, %
                  r14,r14d,r14w,r14b,r15,r15d,r15w,r15b}} % etc.

% header / footer style
\pagestyle{fancy}
\fancyhf{}
\rhead{Systemnahe Informatik SS20}
\lhead{Stefan Schmauch, Anton Lydike}
\rfoot{Seite \thepage}

% define some basic colors
\definecolor{greeen}{RGB}{34,139,34}
\newcommand\red[1]{\textcolor{red}{#1}}
\newcommand\green[1]{\textcolor{greeen}{#1}}
\newcommand\blue[1]{\textcolor{blue}{#1}}


% define a task 
\newcommand\task[2]{\noindent\textbf{Aufgabe #1)\hfill \underline{\,\,\,\,\,\,}\,\,/#2p.}\\}
% and the total points for this sheet
\newcommand\pointsttl[1]{\noindent\textbf{Gesamtpunkte: \hfill \underline{\,\,\,\,\,\,}\,\,/#1p.}\\}


\newcommand\cfgtitle[1]{\title{\vspace{-1.5cm}Übungsblatt #1\\%
\begin{large} Übungsgruppe Pentium \end{large}} \lfoot{Übungsblatt #1}\cfoot{Übungsgruppe Pentium}}
\author{Stefan Schmauch, Anton Lydike}


\usepackage{wasysym}

\cfgtitle{10}
\date{Donnerstag 09.07.2020}

\begin{document}
    \maketitle
    \thispagestyle{fancy}

    \task{1}{6}
    \begin{center}
        \begin{tabular}{|c|c||c|c|}
            \hline 
            virt. Adresse & Zugriff & phys. Adresse & Schutzverl.? \\ \hline
            \texttt{0x000} & Fetch     & \texttt{0x400} & no \\
            \texttt{0x2a0} & Lesen     & \texttt{0x1b0} & size \\
            \texttt{0x1b0} & Schreiben & \texttt{0x130} & read-only \\
            \texttt{0x330} & Lesen     & \texttt{0x0d0} & no \\
            \texttt{0x1c0} & Fetch     & \texttt{0x140} & data-only \\
            \texttt{0x304} & Schreiben & \texttt{0x0a4} & no \\ \hline
        \end{tabular}
    \end{center}

    \task{2}{10}

    \pointsttl{16}
\end{document}


% python code task 1
% access = [0x000, 0x2a0, 0x1b0, 0x330, 0x1c0, 0x304]
% segment_addr = [0x400, 0x080, 0x110, 0x0a0]
% segment_size = [0xff, 0xd0, 0x80, 0x40]
% 
% def segment_num(virt_addr):
%     i    = virt_addr >> 8
%     addr = virt_addr & 0xff
%     if addr > segment_size[i]:
%         return (i, hex(segment_addr[i] + addr), hex(addr), "out")
%     return (i, hex(segment_addr[i] + addr), hex(addr), "in")
% 
% for addr in access:
%     print(f"address {hex(addr)} has solution {segment_num(addr)}")
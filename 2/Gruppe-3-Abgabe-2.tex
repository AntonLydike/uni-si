\documentclass{article}
\usepackage{microtype}
\usepackage[utf8]{inputenc} 
\usepackage[a4paper, total={6in, 9.6in}]{geometry}
\usepackage{enumitem}
\usepackage{amsmath}
\usepackage{amssymb}
\usepackage{fancyhdr}
\usepackage{xcolor}
\usepackage{tikz}
\usepackage{pgfplots}
\usepackage{svg}
\usepackage{graphicx}
\usepackage{listings}

\widowpenalties=4 10000 10000 150 0


% listings sonderzeichen
\lstset{
  literate={ö}{{\"o}}1
           {ä}{{\"a}}1
           {ü}{{\"u}}1
}
\lstdefinelanguage
   [x64]{Assembler}     % add a "x64" dialect of Assembler
   [x86masm]{Assembler} % based on the "x86masm" dialect
   % with these extra keywords:
   {morekeywords={CDQE,CQO,CMPSQ,CMPXCHG16B,JRCXZ,LODSQ,MOVSXD, %
                  POPFQ,PUSHFQ,SCASQ,STOSQ,IRETQ,RDTSCP,SWAPGS, %
                  rax,rdx,rcx,rbx,rsi,rdi,rsp,rbp, %
                  r8,r8d,r8w,r8b,r9,r9d,r9w,r9b, %
                  r10,r10d,r10w,r10b,r11,r11d,r11w,r11b, %
                  r12,r12d,r12w,r12b,r13,r13d,r13w,r13b, %
                  r14,r14d,r14w,r14b,r15,r15d,r15w,r15b}} % etc.

% header / footer style
\pagestyle{fancy}
\fancyhf{}
\rhead{Systemnahe Informatik SS20}
\lhead{Stefan Schmauch, Anton Lydike}
\rfoot{Seite \thepage}

% define some basic colors
\definecolor{greeen}{RGB}{34,139,34}
\newcommand\red[1]{\textcolor{red}{#1}}
\newcommand\green[1]{\textcolor{greeen}{#1}}
\newcommand\blue[1]{\textcolor{blue}{#1}}


% define a task 
\newcommand\task[2]{\noindent\textbf{Aufgabe #1)\hfill \underline{\,\,\,\,\,\,}\,\,/#2p.}\\}
% and the total points for this sheet
\newcommand\pointsttl[1]{\noindent\textbf{Gesamtpunkte: \hfill \underline{\,\,\,\,\,\,}\,\,/#1p.}\\}


\newcommand\cfgtitle[1]{\title{\vspace{-1.5cm}Übungsblatt #1\\%
\begin{large} Übungsgruppe Pentium \end{large}} \lfoot{Übungsblatt #1}\cfoot{Übungsgruppe Pentium}}
\author{Stefan Schmauch, Anton Lydike}


\cfgtitle{2}
\date{Donnerstag 07.05.2020}

\begin{document}
    \maketitle
    \thispagestyle{fancy}

    \task{1}{5+1} 
    \textbf{a)}
    \lstinputlisting[linerange={1-48}]{1.asm}

    \textbf{b)}\\
    Betrachte:
    $$ 12! = 479001600 < 2^{31} = 2147483648 < 2^{32} = 4294967296 < 13! = 6227020800 $$
    $12!$ ist die letzte zahl die wir in einem signed 32 bit register speichern können. 32 bit unsigned lässt $13!$ immer noch nicht zu, da $13! > 2^{32} = 4294967296$

    \vspace{1cm}
    \task{2}{5}
    \lstinputlisting[linerange={1-37}]{2.asm}

    \vspace{1cm}
    \task{3}{2+2+1}
    \begin{center}
        
        \begin{tabular}{l|l}
            RISC  & CISC  \\ \hline 
            0x100 & 0x100 \\ 
            0x104 & 0x104 \\ 
            0x108 & 0x020 \\ 
            0x020 & 0x108 \\ 
            0x10c & 0x30  \\ 
            0x110 & -
        \end{tabular}
    \end{center}

    CISC ist schneller, da es einen RAM Access weniger gibt.

    \vspace{1cm}
    \task{4}{8}
    \begin{center}
        \includegraphics[width=0.7\textwidth]{{4.plot}.pdf}
    \end{center}

    \vspace{1cm}
    \pointsttl{24}
\end{document}

